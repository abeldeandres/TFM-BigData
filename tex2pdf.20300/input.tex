\documentclass[]{article}
\usepackage{lmodern}
\usepackage{amssymb,amsmath}
\usepackage{ifxetex,ifluatex}
\usepackage{fixltx2e} % provides \textsubscript
\ifnum 0\ifxetex 1\fi\ifluatex 1\fi=0 % if pdftex
  \usepackage[T1]{fontenc}
  \usepackage[utf8]{inputenc}
\else % if luatex or xelatex
  \ifxetex
    \usepackage{mathspec}
  \else
    \usepackage{fontspec}
  \fi
  \defaultfontfeatures{Ligatures=TeX,Scale=MatchLowercase}
\fi
% use upquote if available, for straight quotes in verbatim environments
\IfFileExists{upquote.sty}{\usepackage{upquote}}{}
% use microtype if available
\IfFileExists{microtype.sty}{%
\usepackage{microtype}
\UseMicrotypeSet[protrusion]{basicmath} % disable protrusion for tt fonts
}{}
\usepackage[margin=1in]{geometry}
\usepackage{hyperref}
\hypersetup{unicode=true,
            pdftitle={TFM},
            pdfborder={0 0 0},
            breaklinks=true}
\urlstyle{same}  % don't use monospace font for urls
\usepackage{color}
\usepackage{fancyvrb}
\newcommand{\VerbBar}{|}
\newcommand{\VERB}{\Verb[commandchars=\\\{\}]}
\DefineVerbatimEnvironment{Highlighting}{Verbatim}{commandchars=\\\{\}}
% Add ',fontsize=\small' for more characters per line
\usepackage{framed}
\definecolor{shadecolor}{RGB}{248,248,248}
\newenvironment{Shaded}{\begin{snugshade}}{\end{snugshade}}
\newcommand{\KeywordTok}[1]{\textcolor[rgb]{0.13,0.29,0.53}{\textbf{#1}}}
\newcommand{\DataTypeTok}[1]{\textcolor[rgb]{0.13,0.29,0.53}{#1}}
\newcommand{\DecValTok}[1]{\textcolor[rgb]{0.00,0.00,0.81}{#1}}
\newcommand{\BaseNTok}[1]{\textcolor[rgb]{0.00,0.00,0.81}{#1}}
\newcommand{\FloatTok}[1]{\textcolor[rgb]{0.00,0.00,0.81}{#1}}
\newcommand{\ConstantTok}[1]{\textcolor[rgb]{0.00,0.00,0.00}{#1}}
\newcommand{\CharTok}[1]{\textcolor[rgb]{0.31,0.60,0.02}{#1}}
\newcommand{\SpecialCharTok}[1]{\textcolor[rgb]{0.00,0.00,0.00}{#1}}
\newcommand{\StringTok}[1]{\textcolor[rgb]{0.31,0.60,0.02}{#1}}
\newcommand{\VerbatimStringTok}[1]{\textcolor[rgb]{0.31,0.60,0.02}{#1}}
\newcommand{\SpecialStringTok}[1]{\textcolor[rgb]{0.31,0.60,0.02}{#1}}
\newcommand{\ImportTok}[1]{#1}
\newcommand{\CommentTok}[1]{\textcolor[rgb]{0.56,0.35,0.01}{\textit{#1}}}
\newcommand{\DocumentationTok}[1]{\textcolor[rgb]{0.56,0.35,0.01}{\textbf{\textit{#1}}}}
\newcommand{\AnnotationTok}[1]{\textcolor[rgb]{0.56,0.35,0.01}{\textbf{\textit{#1}}}}
\newcommand{\CommentVarTok}[1]{\textcolor[rgb]{0.56,0.35,0.01}{\textbf{\textit{#1}}}}
\newcommand{\OtherTok}[1]{\textcolor[rgb]{0.56,0.35,0.01}{#1}}
\newcommand{\FunctionTok}[1]{\textcolor[rgb]{0.00,0.00,0.00}{#1}}
\newcommand{\VariableTok}[1]{\textcolor[rgb]{0.00,0.00,0.00}{#1}}
\newcommand{\ControlFlowTok}[1]{\textcolor[rgb]{0.13,0.29,0.53}{\textbf{#1}}}
\newcommand{\OperatorTok}[1]{\textcolor[rgb]{0.81,0.36,0.00}{\textbf{#1}}}
\newcommand{\BuiltInTok}[1]{#1}
\newcommand{\ExtensionTok}[1]{#1}
\newcommand{\PreprocessorTok}[1]{\textcolor[rgb]{0.56,0.35,0.01}{\textit{#1}}}
\newcommand{\AttributeTok}[1]{\textcolor[rgb]{0.77,0.63,0.00}{#1}}
\newcommand{\RegionMarkerTok}[1]{#1}
\newcommand{\InformationTok}[1]{\textcolor[rgb]{0.56,0.35,0.01}{\textbf{\textit{#1}}}}
\newcommand{\WarningTok}[1]{\textcolor[rgb]{0.56,0.35,0.01}{\textbf{\textit{#1}}}}
\newcommand{\AlertTok}[1]{\textcolor[rgb]{0.94,0.16,0.16}{#1}}
\newcommand{\ErrorTok}[1]{\textcolor[rgb]{0.64,0.00,0.00}{\textbf{#1}}}
\newcommand{\NormalTok}[1]{#1}
\usepackage{graphicx,grffile}
\makeatletter
\def\maxwidth{\ifdim\Gin@nat@width>\linewidth\linewidth\else\Gin@nat@width\fi}
\def\maxheight{\ifdim\Gin@nat@height>\textheight\textheight\else\Gin@nat@height\fi}
\makeatother
% Scale images if necessary, so that they will not overflow the page
% margins by default, and it is still possible to overwrite the defaults
% using explicit options in \includegraphics[width, height, ...]{}
\setkeys{Gin}{width=\maxwidth,height=\maxheight,keepaspectratio}
\IfFileExists{parskip.sty}{%
\usepackage{parskip}
}{% else
\setlength{\parindent}{0pt}
\setlength{\parskip}{6pt plus 2pt minus 1pt}
}
\setlength{\emergencystretch}{3em}  % prevent overfull lines
\providecommand{\tightlist}{%
  \setlength{\itemsep}{0pt}\setlength{\parskip}{0pt}}
\setcounter{secnumdepth}{0}
% Redefines (sub)paragraphs to behave more like sections
\ifx\paragraph\undefined\else
\let\oldparagraph\paragraph
\renewcommand{\paragraph}[1]{\oldparagraph{#1}\mbox{}}
\fi
\ifx\subparagraph\undefined\else
\let\oldsubparagraph\subparagraph
\renewcommand{\subparagraph}[1]{\oldsubparagraph{#1}\mbox{}}
\fi

%%% Use protect on footnotes to avoid problems with footnotes in titles
\let\rmarkdownfootnote\footnote%
\def\footnote{\protect\rmarkdownfootnote}

%%% Change title format to be more compact
\usepackage{titling}

% Create subtitle command for use in maketitle
\newcommand{\subtitle}[1]{
  \posttitle{
    \begin{center}\large#1\end{center}
    }
}

\setlength{\droptitle}{-2em}
  \title{TFM}
  \pretitle{\vspace{\droptitle}\centering\huge}
  \posttitle{\par}
  \author{}
  \preauthor{}\postauthor{}
  \date{}
  \predate{}\postdate{}


\begin{document}
\maketitle

\subsection{PREPARACION}\label{preparacion}

\subsubsection{FASE I: Clasificación entre: ALIVE, DEATH, NO-DATA y
MD/GR}\label{fase-i-clasificacian-entre-alive-death-no-data-y-mdgr}

\begin{itemize}
\item
  \textbf{En esta fase, se ha realizado una clasificacion de los datos
  dados según 4 resultados finales:}
\item
  Fallecidos o con discapacidades severas (SD-D)
\item
  Con discapacidad moderada o buena recuperacion (MR-GR)
\item
  Vivos (pero sin resultados finales)
\item
  Sin datos
\item
  \textbf{Para este procesado se han tenido en cuenta principalmente las
  siguientes variables:}
\item
  EO\_Outcome
\item
  EO\_Symptoms
\item
  TH\_Outcome
\item
  TH\_Symptoms
\item
  GOS5
\item
  GOS8
\end{itemize}

\paragraph{\texorpdfstring{Cuando las variables de GOS5 y GOS8
\textbf{tienen}
datos}{Cuando las variables de GOS5 y GOS8 tienen datos}}\label{cuando-las-variables-de-gos5-y-gos8-tienen-datos}

\begin{itemize}
\tightlist
\item
  Si las filas ya contenian datos en las columnas de GOS5 y GOS8,
  directamente se han clasificado -segun estas variables-. De lo
  contrario, se ha tenido que analizar las otras variables.
\end{itemize}

\begin{Shaded}
\begin{Highlighting}[]
\KeywordTok{head}\NormalTok{(datos.modelo[,}\KeywordTok{c}\NormalTok{(}\DecValTok{17}\NormalTok{,}\DecValTok{18}\NormalTok{,}\DecValTok{27}\NormalTok{,}\DecValTok{28}\NormalTok{,}\DecValTok{29}\NormalTok{,}\DecValTok{30}\NormalTok{)])}
\end{Highlighting}
\end{Shaded}

\begin{verbatim}
##   EO_Outcome EO_Symptoms TH_Outcome TH_Symptoms GOS5 GOS8
## 1          4           1         NA          NA <NA> <NA>
## 2          4           3         NA          NA <NA>  MD+
## 3          4           2         NA          NA  SD* <NA>
## 4          4           2         NA          NA <NA>  GR+
## 5          4           1         NA          NA <NA> <NA>
## 6          4           2         NA          NA <NA>  SD-
\end{verbatim}

\paragraph{\texorpdfstring{Cuando las variables de GOS5 y GOS8
\textbf{no tienen}
datos}{Cuando las variables de GOS5 y GOS8 no tienen datos}}\label{cuando-las-variables-de-gos5-y-gos8-no-tienen-datos}

\begin{itemize}
\tightlist
\item
  Si las variables de Outcome contenian el valor de 1 (fallecimiento) o
  las variables de Symptoms contenian el valor de 6, directamente esas
  filas del dataset pasaban a clasificarse como fallecidos.
\end{itemize}

\begin{verbatim}
##    EO_Outcome EO_Symptoms TH_Outcome TH_Symptoms GOS5 GOS8 TH_Cause
## 22          1           6         NA          NA <NA> <NA>       NA
## 38          1           6         NA          NA <NA> <NA>       NA
## 50          1           6         NA          NA <NA> <NA>       NA
## 55          1           6         NA          NA <NA> <NA>       NA
## 61          1           6         NA          NA <NA> <NA>       NA
## 85          1           6         NA          NA <NA> <NA>       NA
\end{verbatim}

\begin{itemize}
\tightlist
\item
  Si las variables de Outcome contenian el valor de 4 (alta) y las de
  Symptoms el valor de 1, entonces se han clasificado como ``Vivos (pero
  sin resultados finales)''.
\end{itemize}

\begin{verbatim}
##    EO_Outcome EO_Symptoms TH_Outcome TH_Symptoms GOS5 GOS8 TH_Cause
## 1           4           1         NA          NA <NA> <NA>       NA
## 5           4           1         NA          NA <NA> <NA>       NA
## 18          4           1         NA          NA <NA> <NA>       NA
## 20          4           1         NA          NA <NA> <NA>       NA
## 36          4           1         NA          NA <NA> <NA>       NA
## 51          4           1         NA          NA <NA> <NA>       NA
\end{verbatim}

\begin{itemize}
\tightlist
\item
  Se clasificaran como ``Sin datos'' todas aquellas filas que no
  contengan valores ni en las columnas de Symptomps. Se tienen en cuenta
  los transferidos a otros hospitales.
\end{itemize}

\begin{verbatim}
##      EO_Outcome EO_Symptoms TH_Outcome TH_Symptoms GOS5 GOS8 TH_Cause
## 384          NA          NA         NA          NA <NA> <NA>       NA
## 417          NA          NA         NA          NA <NA> <NA>       NA
## 985          NA          NA         NA          NA <NA> <NA>       NA
## 997          NA          NA         NA          NA <NA> <NA>       NA
## 2270         NA          NA         NA          NA <NA> <NA>       NA
## 2292         NA          NA         NA          NA <NA> <NA>       NA
\end{verbatim}

\begin{itemize}
\tightlist
\item
  Si los Symptoms son de 4 o de 5 (Discapacidad Severa), entonces se
  clasificaran como ``Fallecidos o con discapacidades severas''
\end{itemize}

\begin{verbatim}
##     EO_Outcome EO_Symptoms TH_Outcome TH_Symptoms GOS5 GOS8 TH_Cause
## 160          5           5         NA          NA <NA> <NA>       NA
## 241          5           5         NA          NA <NA> <NA>       NA
## 317          5           5         NA          NA <NA> <NA>       NA
## 336          5           5         NA          NA <NA> <NA>       NA
## 357          5           5         NA          NA <NA> <NA>       NA
## 573          5           5         NA          NA <NA> <NA>       NA
\end{verbatim}

\begin{itemize}
\tightlist
\item
  Asi mismo, si las variables de Outcome contenian el valor de 4 y las
  de Symptoms el valor de 9, significa que el paciente ha sido dado de
  alta, pero no se tiene ningun dato sobre el estado final, por lo
  tanto, se han incluido en la clasificacion de ``Vivos (pero sin
  resultados finales)''.
\end{itemize}

\begin{verbatim}
##      EO_Outcome EO_Symptoms TH_Outcome TH_Symptoms GOS5 GOS8 TH_Cause
## 409           4           9         NA          NA <NA> <NA>       NA
## 1000          4           9         NA          NA <NA> <NA>       NA
## 4859          4           9         NA          NA <NA> <NA>       NA
\end{verbatim}

\begin{itemize}
\tightlist
\item
  Se han visto 3 elementos de NODATA, cuyos pacientes obtienen un estado
  de symptoma 4, por lo que se envia a estado de fallecido, son datos
  anomalos.
\end{itemize}

\begin{verbatim}
##      EO_Outcome EO_Symptoms TH_Outcome TH_Symptoms GOS5 GOS8 TH_Cause
## 52            2           4         NA           4 <NA> <NA>        3
## 699           2           4         NA           4 <NA> <NA>       NA
## 3025          2           4         NA           4 <NA> <NA>        1
\end{verbatim}

\paragraph{DATOS FINALES:}\label{datos-finales}

\begin{itemize}
\tightlist
\item
  Fallecidos o con discapacidades severas
\end{itemize}

\begin{verbatim}
## [1] 3559
\end{verbatim}

\begin{itemize}
\tightlist
\item
  Con discapacidad moderada o buena recuperacion
\end{itemize}

\begin{verbatim}
## [1] 5997
\end{verbatim}

\begin{itemize}
\tightlist
\item
  Vivos (pero sin resultados finales)
\end{itemize}

\begin{verbatim}
## [1] 127
\end{verbatim}

\begin{itemize}
\tightlist
\item
  Sin datos
\end{itemize}

\begin{verbatim}
## [1] 86
\end{verbatim}

\begin{itemize}
\tightlist
\item
  Con NA
\end{itemize}

\begin{verbatim}
## [1] 239
\end{verbatim}

\subsubsection{FASE II: Clasificación entre: ESCANEADOS y NO
ESCANEADOS}\label{fase-ii-clasificacian-entre-escaneados-y-no-escaneados}

\begin{itemize}
\tightlist
\item
  En primer lugar, se han encontrado ciertos datos anomalos, en los que
  aparecen datos escaneados (1) y no tienen los datos del escaner,
  entonces deberiamos ponerlo como no escaneado (2).
\end{itemize}

\begin{verbatim}
##      EO_Head.CT.scan EO_1.or.more.PH EO_Subarachnoid.bleed
## 201                1              NA                    NA
## 314                1              NA                    NA
## 1277               1              NA                    NA
## 3234               1              NA                    NA
## 3687               1              NA                    NA
## 4256               1              NA                    NA
##      EO_Obliteration.3rdVorBC EO_Midline.shift..5mm EO_Non.evac.haem
## 201                        NA                    NA               NA
## 314                        NA                    NA               NA
## 1277                       NA                    NA               NA
## 3234                       NA                    NA               NA
## 3687                       NA                    NA               NA
## 4256                       NA                    NA               NA
##      EO_Evac.haem
## 201            NA
## 314            NA
## 1277           NA
## 3234           NA
## 3687           NA
## 4256           NA
\end{verbatim}

\begin{itemize}
\item
  A continuación se van a clasificar los datos como:
\item
  Escaneados
\end{itemize}

\begin{verbatim}
##    EO_Head.CT.scan EO_1.or.more.PH TH_Head.CT.scan TH_1.or.more.PH outcome
## 2                1               2            <NA>              NA    MDGR
## 7                1               2            <NA>              NA    MDGR
## 8                1               1            <NA>              NA    MDGR
## 14               1               2               1               2       D
## 17               1               1            <NA>              NA    MDGR
## 22               1               2            <NA>              NA       D
\end{verbatim}

\begin{itemize}
\tightlist
\item
  No escaneados
\end{itemize}

\begin{verbatim}
##    EO_Head.CT.scan EO_1.or.more.PH TH_Head.CT.scan TH_1.or.more.PH outcome
## 3                2              NA            <NA>              NA       D
## 4                2              NA            <NA>              NA    MDGR
## 6                2              NA            <NA>              NA       D
## 9                2              NA            <NA>              NA    MDGR
## 10               2              NA            <NA>              NA    MDGR
## 11               2              NA            <NA>              NA    MDGR
\end{verbatim}

\begin{itemize}
\tightlist
\item
  En analisis
\end{itemize}

\begin{verbatim}
##     EO_Head.CT.scan EO_1.or.more.PH TH_Head.CT.scan TH_1.or.more.PH
## 192            <NA>              NA            <NA>              NA
## 227            <NA>              NA            <NA>              NA
## 240            <NA>              NA            <NA>              NA
## 267            <NA>              NA               1               2
## 329            <NA>              NA               1               2
## 354            <NA>              NA               1               1
##     outcome
## 192       D
## 227       D
## 240       D
## 267       D
## 329       D
## 354       D
\end{verbatim}

\begin{itemize}
\tightlist
\item
  Si el Outcome es 2 (el paciente se ha transferido a otro hospital), se
  ha escaneado en dicho hospital (TH\_SCAN) y no se tiene ninguna
  información en los escáneres, se clasificaran como ``En análisis''.
\end{itemize}

\begin{verbatim}
##     EO_Outcome TH_Head.CT.scan TH_1.or.more.PH TH_Subarachnoid.bleed
## 52           2            <NA>              NA                    NA
## 128          2            <NA>              NA                    NA
## 135          2            <NA>              NA                    NA
## 188          2            <NA>              NA                    NA
## 193          2            <NA>              NA                    NA
## 207          2            <NA>              NA                    NA
\end{verbatim}

\begin{itemize}
\tightlist
\item
  Sobre el dataset NOSCANEADO: Si el Outcome es 2 (el paciente se ha
  transferido a otro hospital) y no se ha realizado ningun escaner pero
  si contiene datos en el escaner, entonces se clasificara como
  ``Escaneado''.
\end{itemize}

\begin{verbatim}
##      EO_Outcome TH_Head.CT.scan TH_1.or.more.PH TH_Subarachnoid.bleed
## 201           2               1               2                     2
## 217           2               1               2                     1
## 257           2               1               1                     2
## 314           2               1               1                     2
## 318           2               1               2                     2
## 1184          2               1               2                     2
\end{verbatim}

\begin{itemize}
\tightlist
\item
  Sobre el dataset NOSCANEADO: Nos hemos dado cuenta que existen datos
  anomalos, que contienen varios escaneres, pero sin embargo, no se
  indica como escaneado, son los registros: 2628,3276,3279,8469,8655,
  etc. (En total son 12)
\end{itemize}

\begin{verbatim}
##      EO_Head.CT.scan EO_1.or.more.PH EO_Subarachnoid.bleed
## 2628               2               2                     2
## 3276               2               2                     2
## 3279               2               2                     2
## 3720               2               2                     2
## 7286               2               2                     2
## 8469               2               2                     2
\end{verbatim}

\begin{itemize}
\tightlist
\item
  Sobre el dataset ENANALISIS: Nos hemos dado cuenta de que existen
  datos anomalos. Para las variables de los pacientes que se han
  transferido a otro hospital (TH), existen variables de escaner
  (TH\_Head.CT.scan) que se encuentran vacias, junto con el resto de
  variables del escaner en particular. Por lo tanto, se ha asignado el
  valor de 2 a la variable de escaner (TH\_Head.CT.scan) y se han
  incluido en los escaneados, puesto que en todos ellos, en la variable
  EO\_Head.CT.scan si que existe un valor de 1 (escaneados) y no se han
  encontrado mas anomalias en dichos datos.
\end{itemize}

\begin{verbatim}
##      EO_Head.CT.scan EO_1.or.more.PH EO_Outcome TH_Head.CT.scan
## 681                1               2          2            <NA>
## 1639               1               2          2            <NA>
## 5743               1               2          2            <NA>
## 8434               1               2          2            <NA>
## 8972               1               2          2            <NA>
##      TH_1.or.more.PH TH_Subarachnoid.bleed
## 681               NA                    NA
## 1639              NA                    NA
## 5743              NA                    NA
## 8434              NA                    NA
## 8972              NA                    NA
\end{verbatim}

\begin{itemize}
\tightlist
\item
  Sobre el dataset SCANEADO: Se van a eliminar todas las filas que no
  tengan informacion en el TH\_Major.EC.injury y en el
  EO\_Major.EC.injury
\end{itemize}

\begin{verbatim}
##     EO_Outcome TH_Major.EC.injury
## 76           2                 NA
## 90           2                 NA
## 315          2                 NA
## 361          2                 NA
## 510          2                 NA
## 565          2                 NA
\end{verbatim}

\begin{itemize}
\tightlist
\item
  Sobre el dataset SCANEADO: Comprobamos que las variables: EO\_Cause y
  EO\_Symptoms, no contengan valores nulos.
\end{itemize}

\begin{verbatim}
##     EO_Cause EO_Major.EC.injury
## 177        3                  2
## 211       NA                  1
## 242       NA                  2
## 255        2                  2
## 293       NA                  2
## 321       NA                  1
\end{verbatim}

\paragraph{DATOS FINALES:}\label{datos-finales-1}

\begin{itemize}
\tightlist
\item
  Vivos y escaneados
\end{itemize}

\begin{verbatim}
## [1] 4158
\end{verbatim}

\begin{itemize}
\tightlist
\item
  Vivos y no escaneados
\end{itemize}

\begin{verbatim}
## [1] 1535
\end{verbatim}

\begin{itemize}
\tightlist
\item
  Vivos en análisis
\end{itemize}

\begin{verbatim}
## [1] 304
\end{verbatim}

\begin{itemize}
\tightlist
\item
  Fallecidos y escaneados
\end{itemize}

\begin{verbatim}
## [1] 2830
\end{verbatim}

\begin{itemize}
\tightlist
\item
  Fallecidos y no escaneados
\end{itemize}

\begin{verbatim}
## [1] 439
\end{verbatim}

\begin{itemize}
\tightlist
\item
  Fallecidos en análisis
\end{itemize}

\begin{verbatim}
## [1] 290
\end{verbatim}

\subsubsection{FASE III: Eliminacion y centralizacion de
variables}\label{fase-iii-eliminacion-y-centralizacion-de-variables}

\begin{itemize}
\tightlist
\item
  Se va a centralizar las variables de PUPIL\_REACT\_LEFT y
  PUPIL\_REACT\_RIGHT
\end{itemize}

\begin{verbatim}
##   PUPIL_REACT_LEFT PUPIL_REACT_RIGHT ESTADOESCANER
## 1                1                 1      SCANEADO
## 2                1                 1      SCANEADO
## 3                1                 1      SCANEADO
## 4                1                 1      SCANEADO
## 5                1                 1      SCANEADO
## 6                1                 1      SCANEADO
\end{verbatim}

\begin{itemize}
\tightlist
\item
  Both reactive
\end{itemize}

\begin{verbatim}
## [1] 5664
\end{verbatim}

\begin{itemize}
\tightlist
\item
  No response unilateral
\end{itemize}

\begin{verbatim}
## [1] 497
\end{verbatim}

\begin{itemize}
\tightlist
\item
  No response
\end{itemize}

\begin{verbatim}
## [1] 634
\end{verbatim}

\begin{itemize}
\tightlist
\item
  Unable to assess
\end{itemize}

\begin{verbatim}
## [1] 193
\end{verbatim}

\begin{itemize}
\tightlist
\item
  Ahora vamos a ver si podemos prescindir o aunar las variables de
  EO\_Cause y TH\_Cause. Para ello veremos en que caso, ambas variables
  difieren:
\end{itemize}

\begin{verbatim}
##      EO_Cause TH_Cause
## 2308        1        3
## 2814        3        1
## 3286        2        3
## 4022        3        1
\end{verbatim}

\begin{itemize}
\tightlist
\item
  Como se puede observar, podriamos prescindir de la variable TH\_Cause,
  puesto que recoge la misma información que EO\_Cause.
\end{itemize}


\end{document}
